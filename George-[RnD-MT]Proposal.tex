\documentclass[rnd]{mas_proposal}
% \documentclass[thesis]{mas_proposal}

\usepackage[utf8]{inputenc}
\usepackage{amsmath}
\usepackage{amsfonts}
\usepackage{amssymb}
\usepackage{graphicx}

\title{Project Proposal Title}
\author{Salvin George}
\supervisors{First Supervisor\\Second Supervisor\\Third Supervisor}
\date{Month 20XX}

% \thirdpartylogo{path/to/your/image}

\begin{document}

\maketitle

\pagestyle{plain}

\section{Introduction}

\subsection{Topic of This R\&D Project}
\begin{itemize}
    %\item Provide reasonably detailed description of what you intent to do in your R\&D project.
    %\item You may also discuss the challenges that you have to address.
    %\item Reflect on the profile of the reader and PLEAAAASE, tell a story here and refrain from bombarding the readers with details which they may not be able to appreciate.
    \item Radioactivity in a inevitable factor of life, and it is said that without radioactivity, life wouldnt be possible. The naturally occuring radioactivity heated up the earths core and there by leading to life. [https://www.energy.gov/science/doe-explainsradioactivity]
    \item Since Henri Becquerel and discovered Radioactivity in 1896, the field has progressed alot further and it has become an important part of how the modern world ticks.
    \item Radioactivity is the process in which the radioactive atom trying to reach stability by ejecting particles, or by releasing energy in other forms.
    \item The Radioactive materials can be misused for malicious purposes, either by using it as a weapon or by using it to contaminate the environment.
    \item Because of its potential for misuse and the potential for harm, it is important to find the radioactive sources and to secure them in case of contamination. 
    \item While this approach has been explored in the past, the current methods are computationally expensive, and it needs the searching of the full area to localize the source, thereby making it efficient enough to detect the radioactive sources in a timely manner.
    \item The purpose of this project is to explore the methods to localize the radioactive sources and improving the localization and path planning of UAV, to develop a framework for simulating the radioactive sources realistically, and to evaluate the methods to provide a comparative evaluation of the methods.
    \item While there are challenges of the developing this project mostly through simulations as testing with real radioactive source is difficult to setup, the project aims to explore well established and new methods to provide a comparative evaluation of the methods.
    \item There also exist challenges to handle the particle attenuation and scattering, which would make the approach to the source difficult.
\end{itemize}

\subsection{Relevance of This R\&D Project}
This Research and Developement project holds relevance due to the following reasons:
\begin{itemize}
    % \item Who will benefit from the results of this R\&D project?
    % \item What are the benefits? Quantify the benefits with concrete numbers.
    \item The results of this project will be beneficial to the security agencies and the law enforcement agencies to detect the radioactive sources in a timely manner.
    \item This localization of the radioactive sources can be useful to the nuclear power plants to detect the leakages in the reactor.
    \item Localizing the the radioactive sources prevent the contamination of the environment and the food chain.
    \item Since 1993, there has bee
 \end{itemize}

\section{Related Work}

\subsection{Survey of Related Work}
\begin{itemize}
    \item What have other people done to solve the problem?
    \item You should reference and briefly discuss at least the ``top twelve'' related works
\end{itemize}

\subsection{Limitation and Deficits in the State of the Art}
\begin{itemize}
    \item List the deficits that you have discovered in the related work and explain them such that a person who is not deep into the technical details can still understand them.
    For each deficit, provide at least two references
    \item You should reference and briefly discuss at least the ``top twelve'' related works
\end{itemize}

\section{Problem Statement}
\begin{itemize}
    \item Which of the deficits are you going to solve?
    \item What is your intended approach?
    \item How will you compare you approach with existing approaches?
\end{itemize}

\section{Project Plan}

\subsection{Work Packages}
\emph{Planning is the replacement of randomness by error.} (Einstein). Very much like you would never start a longer journey without a detailed travel plan, you should not start a project without a carefully though out work plan. A work package is a logical decomposition of a larger piece of work into smaller parts following a ``divide and conquer" strategy. It is very specific to the problem that you are going to address. Refrain from a rather generic decomposition. If your work plan looks similar to those of your school mates, which may address completely different problems then you have not thought carefully enough about how you approach the problem. It is ok to have two generic work packages \emph{Literature Study} and \emph{Project Report}. Discuss your work packages in the ASW seminar.

The bare minimum will include the following packages:
\begin{enumerate}
    \item[WP1] Literature Study
    \item[WP2] Method Selection
    \item[WP3] Simulation Framework
    \item[WP4] Implementation and Experimentation
    \item[WPy] Evaluation and Comparison of the implemented approaches
    \item[WPz] Project Report
\end{enumerate}

\subsection{Milestones}
Milestones mark the completion of a certain activity or at least a major achievement in an activity. Milestones are also decision points, where you reflect on what you have achieved and what options you have for continuing your work in case you have not achieved what was planned. Above all, milestones have to be measurable. As above, if your milestones are the same as those of your school mates, then you may not have thought carefully enough about how your project shall progress.

The following milestone are considered to achieve a planned and strategic approach towards the completion of the project:
\begin{enumerate}
    \item[M1] Completing litertaure survey
    \item[M2] Selecting the approaches to be followed
    \item[M3] Completing the implementation of simulation framework
    \item[M4] Completing the implementation of the selection methods
    \item[M5] Evaluation of the methods
    \item[M6] Final Submission
\end{enumerate}

\subsection{Project Schedule}
Include a Gantt chart here. It doesn't have to be detailed, but it should include the milestones you mentioned above.
Make sure to include the writing of your report throughout the whole project, not just at the end.

\begin{figure}[h!]
    \includegraphics[width=\textwidth]{images/rnd_deliverable_timeline}
    \caption{My figure caption}
    \label{fig:myfigure}
\end{figure}

\subsection{Deliverables}

\subsubsection*{Minimum Viable}
\begin{itemize}
    \item Project results required to get a satisfying or sufficient grade.
\end{itemize}

\subsubsection*{Expected}
\begin{itemize}
    \item Project results required to get a good grade.
\end{itemize}

\subsubsection*{Desired}
\begin{itemize}
    \item Project results required to get an excellent grade.
\end{itemize}

Please note that the final grade will not only depend on the results obtained in your work, but also on how you present the results.

\nocite{*}

\bibliographystyle{plainnat} % Use the plainnat bibliography style
\bibliography{bibliography.bib} % Use the bibliography.bib file as the source of references

\end{document}
